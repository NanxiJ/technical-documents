%%%%%%%%%%%%%%%%%%%%%%%%%%%%%%%%%%%%%%%%%
% Beamer Presentation
% LaTeX Template
% Version 1.0 (10/11/12)
%
% This template has been downloaded from:
% http://www.LaTeXTemplates.com
%
% License:
% CC BY-NC-SA 3.0 (http://creativecommons.org/licenses/by-nc-sa/3.0/)
%
%%%%%%%%%%%%%%%%%%%%%%%%%%%%%%%%%%%%%%%%%

%----------------------------------------------------------------------------------------
%	PACKAGES AND THEMES
%----------------------------------------------------------------------------------------

\documentclass{beamer}

\mode<presentation> {

% The Beamer class comes with a number of default slide themes
% which change the colors and layouts of slides. Below this is a list
% of all the themes, uncomment each in turn to see what they look like.

%\usetheme{default}
%\usetheme{AnnArbor}
%\usetheme{Antibes}
%\usetheme{Bergen}
%\usetheme{Berkeley}
%\usetheme{Berlin}
%\usetheme{Boadilla}
%\usetheme{CambridgeUS}
%\usetheme{Copenhagen}
%\usetheme{Darmstadt}
%\usetheme{Dresden}
%\usetheme{Frankfurt}
%\usetheme{Goettingen}
%\usetheme{Hannover}
%\usetheme{Ilmenau}
%\usetheme{JuanLesPins}
%\usetheme{Luebeck}
\usetheme{Madrid}
%\usetheme{Malmoe}
%\usetheme{Marburg}
%\usetheme{Montpellier}
%\usetheme{PaloAlto}
%\usetheme{Pittsburgh}
%\usetheme{Rochester}
%\usetheme{Singapore}
%\usetheme{Szeged}
%\usetheme{Warsaw}

% As well as themes, the Beamer class has a number of color themes
% for any slide theme. Uncomment each of these in turn to see how it
% changes the colors of your current slide theme.

%\usecolortheme{albatross}
%\usecolortheme{beaver}
%\usecolortheme{beetle}
%\usecolortheme{crane}
%\usecolortheme{dolphin}
%\usecolortheme{dove}
%\usecolortheme{fly}
%\usecolortheme{lily}
%\usecolortheme{orchid}
%\usecolortheme{rose}
%\usecolortheme{seagull}
%\usecolortheme{seahorse}
%\usecolortheme{whale}
%\usecolortheme{wolverine}

%\setbeamertemplate{footline} % To remove the footer line in all slides uncomment this line
%\setbeamertemplate{footline}[page number] % To replace the footer line in all slides with a simple slide count uncomment this line

%\setbeamertemplate{navigation symbols}{} % To remove the navigation symbols from the bottom of all slides uncomment this line
}

\usepackage{graphicx} % Allows including images
\usepackage{booktabs} % Allows the use of \toprule, \midrule and \bottomrule in tables
\usepackage{cool}

%----------------------------------------------------------------------------------------
%	TITLE PAGE
%----------------------------------------------------------------------------------------

\title[Fundamental Theorems of Asset Pricing]{A Simple and Intuitive Coverage of \\ The Fundamental Theorems of Asset Pricing} % The short title appears at the bottom of every slide, the full title is only on the title page

\author{Ashwin Rao} % Your name
\institute[Stanford] % Your institution as it will appear on the bottom of every slide, may be shorthand to save space
{
ICME, Stanford University
% Your institution for the title page
}

\date{\today} % Date, can be changed to a custom date

\begin{document}
\begin{frame}
\titlepage % Print the title page as the first slide
\end{frame}

\begin{frame}
\frametitle{Overview} % Table of contents slide, comment this block out to remove it
\tableofcontents % Throughout your presentation, if you choose to use \section{} and \subsection{} commands, these will automatically be printed on this slide as an overview of your presentation
\end{frame}

\section{Simple Setting for Intuitive Understanding}

\begin{frame}
\frametitle{Simple Setting for Intuitive Understanding}
\begin{itemize}
\item Single-period setting (two time points $t=0$ and $t=1$)
\item $t=0$ has a single state (we'll call it ``Spot'' state)
\item $t=1$ has $n$ random states represented by $\Omega = \{\omega_1, \ldots, \omega_n\}$
\item With probability distribution $\mu: \Omega \rightarrow [0,1]$, i.e, $\sum_{i=1}^n \mu(\omega_i) = 1$
\item $m + 1$ fundamental assets $A_0, A_1, \ldots, A_m$
\item Spot Price (at $t=0$) of $A_j$ denoted $S_j^{(0)}$ for all $j = 0, 1, \ldots, m$
\item Price of $A_j$ in state $\omega_i$ denoted $S_j^{(i)}$ for all $j = 0, \ldots, m, i = 1, \ldots, n$
\item All asset prices are assumed to be real numbers, i.e. in $\mathbb{R}$
\item $A_0$ is a special asset known as risk-free asset with $S_0^{(0)}$ normalized to 1
\item $S_0^{(i)}= e^r$ for all $i = 1, \ldots, n$ where $r$ is the constant risk-free rate
\item $e^{-r}$ is the risk-free discount factor to represent ``time value of money''
\end{itemize}
\end{frame}

\section{Portfolios, Arbitrage and Risk-Neutral Measure}
\begin{frame}
\frametitle{Portfolios}
\begin{itemize}
\item A portfolio is a vector $\theta = (\theta_0, \theta_1, \ldots, \theta_m) \in \mathbb{R}^{m+1}$
\item $\theta_j$ is the number of units held in asset $A_j$ for all $j = 0, 1, \ldots, m$
\item Spot Value (at $t=0$) of portfolio $\theta$ denoted $V_{\theta}^{(0)}$ is:
$$V_{\theta}^{(0)} = \sum_{j=0}^m \theta_j \cdot S_j^{(0)}$$
\item Value of portfolio $\theta$ in state $\omega_i$ (at $t=1$) denoted $V_{\theta}^{(i)}$ is:
$$V_{\theta}^{(i)} = \sum_{j=0}^m \theta_j \cdot S_j^{(i)} \mbox{ for all } i = 1, \ldots, n$$
\end{itemize}
\end{frame}


\begin{frame}
\frametitle{Arbitrage Portfolio}
\begin{itemize}
\item An Arbitrage Portfolio $\theta$ is one that ``makes money from nothing''
\item Formally, a portfolio $\theta$ such that:
\begin{itemize}
\item $V_{\theta}^{(0)} \leq 0$
\item $V_{\theta}^{(i)} \geq 0 \mbox{ for all } i = 1, \ldots,n$
\item $\exists i$ in $1, \ldots, n$ such that $\mu(\omega_i) > 0$ and $V_{\theta}^{(i)} > 0$
\end{itemize}
\item So we never end with less value than what we start with and we end with expected value greater than what we start with
\item Arbitrage allows market participants to make infinite returns
\item In an efficient market, arbitrage disappears as participants exploit it
\item Hence, Finance Theory typically assumes ``arbitrage-free'' markets
\end{itemize}
\end{frame}


\begin{frame}
\frametitle{Risk-Neutral Probability Measure}
\begin{itemize}
\item Consider a Probability Distribution $\pi : \Omega \rightarrow [0,1]$ such that 
$$\pi(\omega_i) = 0 \mbox{ if and only if } \mu(\omega_i) = 0 \mbox{ for all } i = 1, \ldots, n$$
\item Then, $\pi$ is a Risk-Neutral Probability Measure if:
\begin{equation}
S_j^{(0)} = e^{-r} \cdot \sum_{i=1}^n \pi(\omega_i) \cdot S_j^{(i)} \mbox{ for all } j = 0, 1, \ldots, m \label{eq:assetdiscexp}
\end{equation}
\item So for each of the $m+1$ assets, the asset spot price (at $t=0$) is the discounted expectation (under $\pi$) of the asset price at $t=1$
\item $\pi$ is an artificial construct to connect expectation of asset prices at $t=1$ to their spot prices by the risk-free discount factor $e^{-r}$
\end{itemize}
\end{frame}

\section{First Fundamental Theorem of Asset Pricing}
\begin{frame}
\frametitle{1st Fundamental Theorem of Asset Pricing (1st FTAP)}
\begin{theorem}
1st FTAP: Our simple setting will not admit arbitrage portfolios if and only if there exists a Risk-Neutral Probability Measure.
\end{theorem}
\begin{itemize}
\item First we prove the easy implication: \\Existence of Risk-Neutral Measure $\Rightarrow$ Arbitrage-free
\item Assume there is a risk-neutral measure $\pi$
\item Then, for each portfolio $\theta = (\theta_0, \theta_1, \ldots, \theta_m)$,
\begin{align*}
V_{\theta}^{(0)} & = \sum_{j=0}^m \theta_j \cdot S_j^{(0)} = \sum_{j=0}^m \theta_j \cdot e^{-r} \cdot \sum_{i=1}^n \pi(\omega_i) \cdot S_j^{(i)} \\
& = e^{-r} \cdot \sum_{i=1}^n \pi(\omega_i) \cdot \sum_{j=0}^m \theta_j \cdot S_j^{(i)} = e^{-r} \cdot \sum_{i=1}^n \pi(\omega_i) \cdot V_{\theta}^{(i)}
\end{align*}
\end{itemize}
\end{frame}

\begin{frame}
\frametitle{1st Fundamental Theorem of Asset Pricing (1st FTAP)}
\begin{itemize}
\item So the portfolio spot value is the discounted expectation (under $\pi$) of the portfolio value at $t=1$
\item For any portfolio $\theta$, if the following two conditions are satisfied:
\begin{itemize}
\item $V_{\theta}^{(i)} \geq 0 \mbox{ for all } i = 1, \ldots,n$
\item $\exists i$ in $1, \ldots, n$ such that $\mu(\omega_i) > 0 (\Rightarrow \pi(\omega_i) > 0)$ and $V_{\theta}^{(i)} > 0$
\end{itemize}
Then, $$V_{\theta}^{(0)} = e^{-r} \cdot \sum_{i=1}^n \pi(\omega_i) \cdot V_{\theta}^{(i)} > 0$$
\item This eliminates the the possibility of arbitrage for any portfolio $\theta$
\item The other implication (Arbitrage-free $\Rightarrow$ Existence of Risk-Neutral Measure) is harder to prove and covered in Appendix 1
\end{itemize}
\end{frame}

\section{Derivatives, Replicating Portfolios and Hedges}

\begin{frame}
\frametitle{Derivatives, Replicating Portfolios and Hedges}
\begin{itemize}
\item A Derivative $D$ (in this simple setting) is a vector payoff at $t=1$:
$$(V_D^{(1)}, V_D^{(2)}, \ldots, V_D^{(n)})$$
where $V_D^{(i)}$ is the payoff of the derivative in state $\omega_i$ for all $i = 1, \ldots, n$
\item Portfolio $\theta \in \mathbb{R}^{m+1}$ is a {\em Replicating Portfolio} for derivative $D$ if:
\begin{equation}
V_D^{(i)} = \sum_{j=0}^m \theta_j \cdot S_j^{(i)} \mbox{ for all } i = 1, \ldots, n \label{eq:endreplport}
\end{equation}
\item The negatives of the components $(\theta_0, \theta_1, \ldots, \theta_m)$ are known as the {\em hedges} for $D$ since they can be used to offset 
the risk in the payoff of $D$ at $t=1$
\end{itemize}
\end{frame}

\section{Second Fundamental Theorem of Asset Pricing}

\begin{frame}
\frametitle{2nd Fundamental Theorem of Asset Pricing (2nd FTAP)}
An arbitrage-free market is said to be {\em Complete} if every derivative in the market has a replicating portfolio.
\begin{theorem}
2nd FTAP: A market is Complete in our simple setting if and only if there is a unique risk-neutral probability measure.
\end{theorem}
Proof in Appendix 2. Together, the FTAPs classify markets into:
\begin{enumerate} 
\item Complete (arbitrage-free) market $\Leftrightarrow$ Unique risk-neutral measure
\item Market with arbitrage $\Leftrightarrow$ No risk-neutral measure
\item Incomplete (arbitrage-free) market $\Leftrightarrow$ Multiple risk-neutral measures
\end{enumerate}
The next topic is derivatives pricing that is based on the concepts of {\em replication of derivatives} and {\em risk-neutral measures}, and so is tied to the concepts of {\em arbitrage} and {\em completeness}.
\end{frame}

\section{Derivatives Pricing}

\begin{frame}
\frametitle{Positions involving a Derivative}
\begin{itemize}
\item Before getting into Derivatives Pricing, we need to define a {\em Position}
\item We define a {\em Position} involving a derivative $D$ as the combination of holding some units in $D$ and some units in $A_0, A_1, \ldots, A_m$
\item {\em Position} is an extension of the Portfolio concept including a derivative
\item Formally denoted as $\gamma_D = (\alpha, \theta_0, \theta_1, \ldots, \theta_m) \in \mathbb{R}^{m+2}$
\item $\alpha$ is the units held in derivative $D$
\item $\theta_j$ is the units held in $A_j$ for all $j = 0, 1, \ldots, m$
\item Extend the definition of Portfolio Value to Position Value 
\item Extend the definition of Arbitrage Portfolio to Arbitrage Position
\end{itemize}
\end{frame}

\begin{frame}
\frametitle{Derivatives Pricing: Elimination of candidate prices}
\begin{itemize}
\item We will consider candidate prices (at $t=0$) for a derivative $D$
\item Let $	\theta = (\theta_0, \theta_1, \ldots, \theta_m)$ be a replicating portfolio for $D$
\item Consider the candidate price $\sum_{j=0}^m \theta_j \cdot S_j^{(0)} - x$ for $D$ for any $x > 0$
\item Position $(1, -\theta_0 + x, -\theta_1, \ldots, -\theta_m)$ has value $x \cdot e^r > 0$ in each of the states at $t=1$
\item But this position has spot ($t=0$) value of 0, which means this is an Arbitrage Position, rendering this candidate price invalid
\item Consider the candidate price $\sum_{j=0}^m \theta_j \cdot S_j^{(0)} + x$ for $D$ for any $x > 0$
\item Position $(-1, \theta_0 + x, \theta_1, \ldots, \theta_m)$ has value $x \cdot e^r > 0$ in each of the states at $t=1$
\item But this position has spot ($t=0$) value of 0, which means this is an Arbitrage Position, rendering this candidate price invalid
\item So every candidate price for $D$ other than $\sum_{j=0}^m \theta_j \cdot S_j^{(0)}$ is invalid
\end{itemize}
\end{frame}

\begin{frame}
\frametitle{Derivatives Pricing: Remaining candidate price}
\begin{itemize}
\item Having eliminated various candidate prices for $D$, we now aim to {\em establish} the remaining candidate price:
\begin{equation}
V_D^{(0)} = \sum_{j=0}^m \theta_j \cdot S_j^{(0)} \label{eq:remcandidateprice}
\end{equation}
where $\theta = (\theta_0, \theta_1, \ldots, \theta_m)$ is a replicating portfolio for $D$
\item To eliminate prices, our only assumption was that $D$ can be replicated
\item This can happen in a complete market or in an arbitrage market
\item To establish remaining candidate price $V_D^{(0)}$, we need to assume market is complete, i.e., there is a unique risk-neutral measure $\pi$
\item Candidate price $V_D^{(0)}$ can be expressed as the discounted expectation (under $\pi$) of the payoff of $D$ at $t=1$, i.e.,
\begin{equation}
V_D^{(0)} = \sum_{j=0}^m \theta_j \cdot e^{-r} \cdot \sum_{i=1}^n \pi(\omega_i) \cdot S_j^{(i)} = e^{-r} \cdot \sum_{i=1}^n \pi(\omega_i) \cdot V_D^{(i)} \label{eq:derivdiscexp}
\end{equation}
\end{itemize}
\end{frame}

\begin{frame}
\frametitle{Derivatives Pricing: Establishing remaining candidate price}
\begin{itemize}
\item Now consider an {\em arbitrary portfolio} $\beta = (\beta_0, \beta_1, \ldots, \beta_m)$
\item Define a position $\gamma_D = (\alpha, \beta_0, \beta_1, \ldots, \beta_m)$
\item Spot Value (at $t=0$) of position $\gamma_D$ denoted $V_{\gamma_D}^{(0)}$ is:
\begin{equation}
V_{\gamma_D}^{(0)} = \alpha \cdot V_D^{(0)} + \sum_{j=0}^m \beta_j \cdot S_j^{(0)} \label{eq:startpositionval}
\end{equation}
where $V_D^{(0)}$ is the remaining candidate price
\item Value of position $\gamma_D$ in state $\omega_i$ (at $t=1$), denoted $V_{\gamma_D}^{(i)}$, is:
\begin{equation}
V_{\gamma_D}^{(i)} = \alpha \cdot V_D^{(i)} + \sum_{j=0}^m \beta_j \cdot S_j^{(i)} \mbox{ for all } i = 1, \ldots, n \label{eq:endpositionval}
\end{equation}
\item Combining the linearity in equations (\ref{eq:assetdiscexp}), (\ref{eq:derivdiscexp}), (\ref{eq:startpositionval}), (\ref{eq:endpositionval}), we get:
\begin{equation}
V_{\gamma_D}^{(0)} = e^{-r} \cdot \sum_{i=1}^n \pi(\omega_i) \cdot V_{\gamma_D}^{(i)} \label{eq:positiondiscexp}
\end{equation}
\end{itemize}
\end{frame}

\begin{frame}
\frametitle{Derivatives Pricing: Establishing remaining candidate price}
\begin{itemize}
\item So the position spot value is the discounted expectation (under $\pi$) of the position value at $t=1$
\item For any $\gamma_D$ (containing any arbitrary portfolio $\beta$) and with $V_D^{(0)}$ as the candidate price for $D$, if the following two conditions are satisfied:
\begin{itemize}
\item $V_{\gamma_D}^{(i)} \geq 0 \mbox{ for all } i = 1, \ldots,n$
\item $\exists i$ in $1, \ldots, n$ such that $\mu(\omega_i) > 0 (\Rightarrow \pi(\omega_i) > 0)$ and $V_{\gamma_D}^{(i)} > 0$
\end{itemize}
Then, $$V_{\gamma_D}^{(0)} = e^{-r} \cdot \sum_{i=1}^n \pi(\omega_i) \cdot V_{\gamma_D}^{(i)} > 0$$
\item This eliminates arbitrage possibility for remaining candidate price $V_D^{(0)}$
\item So we have eliminated all prices other than $V_D^{(0)}$, and we have established the price $V_D^{(0)}$, proving that it should be {\em the} price of $D$
\item The above arguments assumed a complete market, but what about an incomplete market or a market with arbitrage?
\end{itemize}
\end{frame}

\begin{frame}
\frametitle{Incomplete Market (Multiple Risk-Neutral Measures)}
\begin{itemize}
\item Recall: Incomplete market means some derivatives can't be replicated
\item Absence of replicating portfolio precludes usual arbitrage arguments
\item 2nd FTAP says there are multiple risk-neutral measures
\item So, multiple derivative prices (each consistent with no-arbitrage)
\item {\em Superhedging} (outline in Appendix 3) provides bounds for the prices
\item But often these bounds are not tight and so, not useful in practice
\item The alternative approach is to identify hedges that maximize Expected Utility of the derivative together with the hedges
\item For an appropriately chosen market/trader \href{https://github.com/coverdrive/technical-documents/blob/master/finance/cme241/UtilityTheoryForRisk.pdf}{\underline{\textcolor{blue}{Utility function}}} 
\item Utility function is a specification of reward-versus-risk preference that effectively chooses the risk-neutral measure and (hence, Price)
\item We outline the Expected Utility approach in Appendix 4
\end{itemize}
\end{frame}


\begin{frame}
\frametitle{Multiple Replicating Portfolios (Arbitrage Market)}
\begin{itemize}
\item Assume there are replicating portfolios $\alpha$ and $\beta$ for $D$
with $$\sum_{j=0}^m \alpha_j \cdot S_j^{(0)} -  \sum_{j=0}^m \beta_j \cdot S_j^{(0)} = x > 0$$
\item Consider portfolio $\theta = (\beta_0 - \alpha_0 + x, \beta_1 - \alpha_1, \ldots, \beta_m - \alpha_m)$
$$V_{\theta}^{(0)} = \sum_{j=0}^m (\beta_j - \alpha_j) \cdot S_j^{(0)} + x \cdot S_0^{(0)} = -x + x = 0$$
$$V_{\theta}^{(i)} = \sum_{j=0}^m (\beta_j - \alpha_j) \cdot S_j^{(i)} + x \cdot S_0^{(i)} = x \cdot e^r > 0 \mbox{ for all } i = 1, \ldots, n$$
\item So $\theta$ is an Arbitrage Portfolio $\Rightarrow$ market with no risk-neutral measure
\item Also note from previous elimination argument that every candidate price other than $\sum_{j=0}^m \alpha_j \cdot S_j^{(0)}$ is invalid and every candidate price other than $\sum_{j=0}^m \beta_j \cdot S_j^{(0)}$ is invalid, so $D$ has no valid price at all
 \end{itemize}
 \end{frame}

\section{Examples}

\begin{frame}
\frametitle{Market with 2 states and 1 Risky Asset}
\begin{itemize}
\item Consider a market with $m = 1$ and $n = 2$
\item Assume $S_1^{(1)} < S_1^{(2)}$
\item No-arbitrage requires $S_1^{(1)} \leq S_1^{(0)} \cdot e^r \leq S_1^{(2)}$
\item Assuming absence of arbitrage and invoking 1st FTAP, there exists a risk-neutral probability measure $\pi$ such that:
$$S_1^{(0)} = e^{-r} \cdot (\pi(\omega_1) \cdot S_1^{(1)} + \pi(\omega_2) \cdot S_1^{(2)})$$
$$\pi(\omega_1) + \pi(\omega_2) = 1$$
\item This implies:
$$\pi(\omega_1) = \frac {S_1^{(2)} - S_1^{(0)} \cdot e^r} {S_1^{(2)} - S_1^{(1)}}$$
$$\pi(\omega_2) = \frac {S_1^{(0)} \cdot e^r - S_1^{(1)}} {S_1^{(2)} - S_1^{(1)}}$$
\end{itemize}
\end{frame}

\begin{frame}
\frametitle{Market with 2 states and 1 Risky Asset (continued)}
\begin{itemize}
\item We can use these probabilities to price a derivative $D$ as:
$$V_D^{(0)} = e^{-r} \cdot (\pi(\omega_1) \cdot V_D^{(1)} + \pi(\omega_2) \cdot V_D^{(2)})$$
\item Now let us try to form a replicating portfolio $(\theta_0, \theta_1)$ for $D$
$$V_D^{(1)} = \theta_0 \cdot e^r + \theta_1 \cdot S_1^{(1)}$$
$$V_D^{(2)} = \theta_0 \cdot e^r + \theta_1 \cdot S_1^{(2)}$$ 
\item Solving this yields Replicating  Portfolio $(\theta_0, \theta_1)$ as follows:
$$\theta_0 = e^{-r} \cdot \frac {V_D^{(1)} \cdot S_1^{(2)} - V_D^{(2)} \cdot S_1^{(1)}} {S_1^{(2)} - S_1^{(1)}} \mbox{ and } \theta_1 = \frac {V_D^{(2)} - V_D^{(1)}} {S_1^{(2)} - S_1^{(1)}}$$
\item This means this is a Complete Market
\item Note that the derivative price can also be expressed as:
$$V_D^{(0)} = \theta_0 + \theta_1 \cdot S_1^{(0)}$$ 
\end{itemize}
\end{frame}

\begin{frame}
\frametitle{Market with 3 states and 1 Risky Asset}
\begin{itemize}
\item Consider a market with $m = 1$ and $n = 3$
\item Assume $S_1^{(1)} < S_1^{(2)} < S_1^{(3)}$
\item No-arbitrage requires $S_1^{(1)} \leq S_1^{(0)} \cdot e^r \leq S_1^{(3)}$
\item Assuming absence of arbitrage and invoking 1st FTAP, there exists a risk-neutral probability measure $\pi$ such that:
$$S_1^{(0)} = e^{-r} \cdot (\pi(\omega_1) \cdot S_1^{(1)} + \pi(\omega_2) \cdot S_1^{(2)} + \pi(\omega_3) \cdot S_1^{(3)})$$
$$\pi(\omega_1) + \pi(\omega_2) + \pi(\omega_3) = 1$$
\item 2 equations \& 3 variables $\Rightarrow$ multiple solutions for $\pi$
\item Each of these solutions for $\pi$ provides a valid price for a derivative $D$
$$V_D^{(0)} = e^{-r} \cdot (\pi(\omega_1) \cdot V_D^{(1)} + \pi(\omega_2) \cdot V_D^{(2)} + \pi(\omega_3) \cdot V_D^{(3)})$$
\end{itemize}
\end{frame}

\begin{frame}
\frametitle{Market with 3 states and 1 Risky Asset (continued)}
\begin{itemize}
\item Now let us try to form a replicating portfolio $(\theta_0, \theta_1)$ for $D$
$$V_D^{(1)} = \theta_0 \cdot e^r + \theta_1 \cdot S_1^{(1)}$$
$$V_D^{(2)} = \theta_0 \cdot e^r + \theta_1 \cdot S_1^{(2)}$$ 
$$V_D^{(3)} = \theta_0 \cdot e^r + \theta_1 \cdot S_1^{(3)}$$
\item 3 equations \& 2 variables $\Rightarrow$ no replication for {\em some} $D$
\item This means this is an Incomplete Market 
\item Don't forget that we have multiple risk-neutral probability measures
\item Meaning we have multiple valid prices for derivatives
\end{itemize}
\end{frame}

\begin{frame}
\frametitle{Market with 2 states and 2 Risky Assets}
\begin{itemize}
\item Consider a market with $m = 2$ and $n = 3$
\item Assume $S_1^{(1)} < S_1^{(2)}$ and $S_2^{(1)} < S_2^{(2)}$
\item Let us try to determine a risk-neutral probability measure $\pi$:
$$S_1^{(0)} = e^{-r} \cdot (\pi(\omega_1) \cdot S_1^{(1)} + \pi(\omega_2) \cdot S_1^{(2)})$$
$$S_2^{(0)} = e^{-r} \cdot (\pi(\omega_1) \cdot S_2^{(1)} + \pi(\omega_2) \cdot S_2^{(2)})$$
$$\pi(\omega_1) + \pi(\omega_2) = 1$$
\item 3 equations \& 2 variables $\Rightarrow$ no risk-neutral measure $\pi$
\item Let's try to form a replicating portfolio $(\theta_0, \theta_1, \theta_2)$ for a derivative $D$
$$V_D^{(1)} = \theta_0 \cdot e^r + \theta_1 \cdot S_1^{(1)} + \theta_2 \cdot S_2^{(1)}$$
$$V_D^{(2)} = \theta_0 \cdot e^r + \theta_1 \cdot S_1^{(2)} + \theta_2 \cdot S_2^{(2)}$$
\end{itemize}
\end{frame}

\begin{frame}
\frametitle{Market with 2 states and 2 Risky Assets (continued)}
\begin{itemize}
\item 2 equations \& 3 variables $\Rightarrow$ multiple replicating portfolios
\item Each such replicating portfolio yields a price for $D$ as:
$$V_D^{(0)} = \theta_0 + \theta_1 \cdot S_1^{(0)} + \theta_2 \cdot S_2^{(0)}$$
\item Select two such replicating portfolios with different $V_D^{(0)}$
\item Combination of these replicating portfolios is an Arbitrage Portfolio
\begin{itemize}
\item They cancel off each other's price in each $t=1$ states
\item They have a combined negative price at $t=0$
\end{itemize}
\item So this is a market that admits arbitrage (no risk-neutral measure)
\end{itemize}
\end{frame}

\section{Summary and General Theory}

\begin{frame}
\frametitle{Summary}
3 cases:
\begin{enumerate}
\item {\em Complete market}
\begin{itemize}
\item Unique replicating portfolio for derivatives
\item Unique risk-neutral measure, meaning we have unique derivatives prices
\end{itemize}
\item {\em Arbitrage-free but incomplete market}
\begin{itemize}
\item Not all derivatives can be replicated
\item Multiple risk-neutral measures, meaning we can have multiple valid prices for derivatives
\end{itemize}
\item {\em Market with Arbitrage}
\begin{itemize}
\item Derivatives have multiple replicating portfolios (that when combined causes arbitrage)
\item No risk-neutral measure, meaning derivatives cannot be priced
\end{itemize}
\end{enumerate}
\end{frame}

\begin{frame}
\frametitle{General Theory for Derivatives Pricing}
\begin{itemize}
\item The theory for our simple setting extends nicely to the general setting
\item Instead of $t=0,1$, we consider $t=0, 1, \ldots, T$
\item The model is a ``recombining tree'' of state transitions across time
\item The idea of Arbitrage applies over multiple time periods
\item Risk-neutral measure for each state at each time period
\item Over multiple time periods, we need a {\em Dynamic Replicating Portfolio} to rebalance asset holdings (``self-financing trading strategy'')
\item We obtain prices and replicating portfolio at each time in each state
\item By making time period smaller and smaller, the model turns into a stochastic process (in continuous time)
\item Classical Financial Math theory based on stochastic calculus but has essentially the same ideas we developed for our simple setting
\end{itemize}
\end{frame}

\appendix
\begin{frame}
\frametitle{Appendix 1: Arbitrage-free $\Rightarrow \exists$ a risk-neutral measure}
\begin{itemize}
\item We will prove that if a risk-neutral probability measure doesn't exist, there exists an arbitrage portfolio
\item Let $\mathbb{V} \subset \mathbb{R}^m$ be the set of vectors $(s_1, \ldots, s_m)$ such that
$$s_j = e^{-r} \cdot \sum_{i=1}^n \mu(\omega_i) \cdot S_j^{(i)} \mbox{ for all } j = 1, \ldots, m$$
spanning over all possible probability distributions $\mu: \Omega \rightarrow [0,1]$
\item $\mathbb{V}$ is a bounded, closed, convex polytope in $\mathbb{R}^m$
\item If a risk-neutral measure doesn't exist, $(S_1^{(0)}, \ldots, S_m^{(0)}) \not\in \mathbb{V}$
\item Hyperplane Separation Theorem implies that there exists a non-zero vector $(\theta_1, \ldots, \theta_m)$ such that
for any $v = (v_1, \ldots, v_m) \in \mathbb{V}$,
$$\sum_{j=1}^m \theta_j \cdot v_j > \sum_{j=1}^m \theta_j \cdot S_j^{(0)}$$
\end{itemize}
\end{frame}

\begin{frame}
\frametitle{Appendix 1: Arbitrage-free $\Rightarrow \exists$ a risk-neutral measure}
\begin{itemize}
\item In particular, consider vectors $v$ corresponding to the corners of $\mathbb{V}$, those for which the full probability
 mass is on a particular $\omega_i \in \Omega$, i.e.,
 $$\sum_{j=1}^m \theta_j \cdot (e^{-r} \cdot S_j^{(i)}) > \sum_{j=1}^m \theta_j \cdot S_j^{(0)} \mbox{ for all } i = 1, \ldots, n$$
 \item Choose a $\theta_0 \in \mathbb{R}$ such that:
 $$\sum_{j=1}^m \theta_j \cdot (e^{-r} \cdot S_j^{(i)}) > -\theta_0 > \sum_{j=1}^m \theta_j \cdot S_j^{(0)} \mbox{ for all } i = 1, \ldots, n$$
 \item Therefore,
 $$e^{-r} \cdot \sum_{j=0}^m \theta_j \cdot S_j^{(i)} > 0 > \sum_{j=0}^m \theta_j \cdot S_j^{(0)} \mbox{ for all } i = 1, \ldots, n$$
 \item This means $(\theta_0, \theta_1, \ldots, \theta_m)$ is an arbitrage portfolio
 \qed
\end{itemize}
\end{frame}

\appendix
\begin{frame}
\frametitle{Appendix 2: Proof of 2nd FTAP}
\begin{itemize}
\item We will first prove that in an arbitrage-free market, if every derivative has a replicating portfolio, there is a unique risk-neutral measure $\pi$
\item We define $n$ special derivatives (known as {\em Arrow-Debreu securities}), one for each random state in $\Omega$ at $t=1$
\item We define the time $t=1$ payoff of {\em Arrow-Debreu security} $D_k$ (for each of $k = 1, \ldots, n$) in state $\omega_i$ as $\mathbb{I}_{i=k}$ for all $i = 1, \ldots, n$.
\item Since each derivative has a replicating portfolio, let $\theta^{(k)} = (\theta_0^{(k)}, \theta_1^{(j)}, \ldots, \theta_m^{(k)})$ be the replicating 
portfolio for $D_k$.
\item With usual no-arbitrage argument, the price (at $t=0$) of $D_k$ is 
$$\sum_{j=0}^m \theta_j^{(k)} \cdot S_j^{(0)} \mbox{ for all } k = 1, \ldots, n$$
\end{itemize}
\end{frame}

\begin{frame}
\frametitle{Appendix 2: Proof of 2nd FTAP}
\begin{itemize}
\item Now let us try to solve for an unknown risk-neutral probability measure $\pi : \Omega \rightarrow [0, 1]$, given the above prices for $D_k, k = 1, \ldots, n$
$$e^{-r} \cdot \sum_{i=1}^n \pi(\omega_i) \cdot \mathbb{I}_{i=k} = e^{-r} \cdot \pi(\omega_k) = \sum_{j=0}^m \theta_j^{(k)} \cdot S_j^{(0)} \mbox{ for all } k = 1, \ldots, n$$
$$\Rightarrow \pi(\omega_k) = e^r \cdot \sum_{j=0}^m \theta_j^{(k)} \cdot S_j^{(0)} \mbox{ for all } k = 1, \ldots, n$$
\item This yields a unique solution for the risk-neutral probability measure $\pi$
\item Next, we prove the other direction of the 2nd FTAP
\item  To prove: if there exists a risk-neutral measure $\pi$ and if there exists a derivative $D$ 
with no replicating portfolio, we can construct a risk-neutral measure different than $\pi$
\end{itemize}
\end{frame}

\begin{frame}
\frametitle{Appendix 2: Proof of 2nd FTAP}
\begin{itemize}
\item Consider the following vectors in the vector space $\mathbb{R}^n$
$$v = (V_D^{(1)}, \ldots, V_D^{(n)}) \mbox{ and } s_j = (S_j^{(1)}, \ldots, S_j^{(n)}) \mbox{ for all } j = 0, 1, \ldots, m$$
\item Since $D$ does not have a replicating portfolio, $v$ is not in the span of $s_0, s_1, \ldots, s_m$, which 
means $s_0, s_1, \ldots, s_m$ do not span $\mathbb{R}^n$
\item Hence $\exists$ a non-zero vector $u = (u_1, \ldots, u_n) \in \mathbb{R}^n$ orthogonal to each of $s_0, s_1, \ldots, s_m$, i.e.,
\begin{equation}
\sum_{i=1}^n u_i \cdot S_j^{(i)} = 0 \mbox{ for all } j = 0,1, \ldots, n \label{eq:orthogonal}
\end{equation}
\item Note that $S_0^{(i)} = e^r$ for all $i = 1, \ldots, n$ and so,
\begin{equation}
\sum_{i=1}^n u_i = 0 \label{eq:partitionunity}
\end{equation}
\end{itemize}
\end{frame}

\begin{frame}
\frametitle{Appendix 2: Proof of 2nd FTAP}
\begin{itemize}
\item Define $\pi' : \Omega \rightarrow \mathbb{R}$ as follows (for some $\epsilon > 0 \in \mathbb{R})$:
\begin{equation}
\pi'(\omega_i) = \pi(\omega_i) + \epsilon \cdot u_i \mbox{ for all } i = 1, \ldots, n \label{eq:newmeasure}
\end{equation}
\item To establish $\pi'$ as a risk-neutral measure different than $\pi$, note:
\begin{itemize}
\item Since $\sum_{i=1}^n \pi(\omega_i) = 1$ and since $\sum_{i=1}^n u_i = 0$, $\sum_{i=1}^n \pi'(\omega_i) = 1$
\item Construct $\pi'(\omega_i) > 0$ for each $i$ where $\pi(\omega_i) > 0$ by making $\epsilon > 0$ sufficiently small, and set $\pi'(\omega_i) = 0$ for each $i$ 
where $\pi(\omega_i) = 0$
\item From Eq (\ref{eq:orthogonal}) and Eq (\ref{eq:newmeasure}), we derive:
$$\sum_{i=1}^n \pi'(\omega_i) \cdot S_j^{(i)} = \sum_{i=1}^n \pi(\omega_i) \cdot S_j^{(i)} = e^r \cdot S_j^{(0)} \mbox{ for all } j = 0, 1, \ldots, m$$
\end{itemize}
\qed
\end{itemize}
\end{frame}

\begin{frame}
\frametitle{Appendix 3: Superhedging}
\begin{itemize}
\item Superhedging is a technique to price in incomplete markets
\item Where one cannot replicate \& there are multiple risk-neutral measures
\item The idea is to create a portfolio of fundamental assets whose Value {\em dominates} the derivative payoff in {\em all} states at $t=1$
\item Superhedge Price is the smallest possible Portfolio Spot ($t=0$) Value among all such Derivative-Payoff-Dominating portfolios
\item This is a constrained linear optimization problem:
\begin{equation}
\min_{\theta} \sum_{j=0}^m \theta_j \cdot S_j^{(0)} \mbox{ such that } \sum_{j=0}^m \theta_j \cdot S_j^{(i)} \geq V_D^{(i)} \mbox{ for all } i = 1, \ldots, n \label{eq:superhedging}
\end{equation}
\item Let $\theta^* = (\theta_0^*, \theta_1^*, \ldots, \theta_m^*)$ be the solution to Equation (\ref{eq:superhedging})
\item Let $SP$ be the Superhedge Price $\sum_{j=0}^m \theta_j^* \cdot S_j^{(0)}$
\end{itemize}
\end{frame}

\begin{frame}
\frametitle{Appendix 3: Superhedging}
\begin{itemize}
\item Establish feasibility and define Lagrangian $J(\theta, \lambda)$
$$J(\theta, \lambda) = \sum_{j=0}^m \theta_j \cdot S_j^{(0)} + \sum_{i=1}^n \lambda_i \cdot (V_D^{(i)} - \sum_{j=0}^m \theta_j \cdot S_j^{(i)})$$
\item So there exists $\lambda = (\lambda_1, \ldots, \lambda_n)$ that satisfy these KKT conditions:
$$\lambda_i \geq 0 \mbox{ for all } i = 1, \ldots, n$$
$$\lambda_i \cdot (V_D^{(i)} - \sum_{j=0}^m \theta_j^* \cdot S_j^{(i)}) = 0 \mbox{ for all } i = 1, \ldots, n $$
$$\nabla_{\theta} J(\theta^*, \lambda) = 0 \Rightarrow S_j^{(0)} = \sum_{i=1}^n \lambda_i \cdot S_j^{(i)} \mbox{ for all } j = 0, 1, \ldots, m$$
\end{itemize}
\end{frame}

\begin{frame}
\frametitle{Appendix 3: Superhedging}
\begin{itemize}
\item This implies $\lambda_i = e^{-r} \cdot \pi(\omega_i)$ for all $i = 1, \ldots, n$ for a risk-neutral probability measure $\pi : \Omega \rightarrow [0,1]$ ($\lambda$ is ``discounted probabilities'')
\item Define Lagrangian Dual $L(\lambda) = \inf_{\theta} J(\theta, \lambda)$. Then, Superhedge Price
$$SP = \sum_{j=0}^m \theta_j^* \cdot S_j^{(0)} = \sup_{\lambda} L(\lambda) = \sup_{\lambda} \inf_{\theta} J(\theta, \lambda)$$
\item Complementary Slackness and some linear algebra over the space of risk-neutral measures $\pi : \Omega \rightarrow [0,1]$ enables us to argue that:
$$SP = \sup_{\pi} \{e^{-r} \cdot \sum_{i=1}^n \pi(\omega_i) \cdot V_D^{(i)}\}$$
\end{itemize}
\end{frame}

\begin{frame}
\frametitle{Appendix 3: Superhedging}
\begin{itemize}
\item Likewise, the {\em Subhedging} price $SB$ is defined as:
\begin{equation}
\max_{\theta} \sum_{j=0}^m \theta_j \cdot S_j^{(0)} \mbox{ such that } \sum_{j=0}^m \theta_j \cdot S_j^{(i)} \leq V_D^{(i)} \mbox{ for all } i = 1, \ldots, n
\end{equation}
\item Likewise arguments enable us to establish:
$$SB = \inf_{\pi} \{e^{-r} \cdot \sum_{i=1}^n \pi(\omega_i) \cdot V_D^{(i)}\}$$
\item This gives a lower bound of $SB$ and an upper bound of $SP$, meaning:
\begin{itemize}
\item A price outside these bounds leads to an arbitrage 
\item Valid prices must be established within these bounds
\end{itemize}
\end{itemize}
\end{frame}

\begin{frame}
\frametitle{Appendix 4: Maximization of Expected Utility}
\begin{itemize}
\item {\em Maximization of Expected Utility} is a technique to establish pricing and hedging in incomplete markets
\item Based on a concave Utility function $U : \mathbb{R} \rightarrow \mathbb{R}$ applied to the Value in each state $\omega_i, i = 1, \ldots n$, at $t=1$
\item An example: $U(x) = \frac {-e^{-ax}} {a}$ where $a \in \mathbb{R}$ is the degree of risk-aversion
\item Let the real-world probabilities be given by $\mu: \Omega \rightarrow [0,1]$
\item Denote $V_D = (V_D^{(1)}, \ldots, V_D^{(n)})$ as the payoff of Derivative $D$ at $t=1$
\item Let $x$ be the candidate price for $D$, which means receiving cash of $-x$ (at $t=0$) as compensation for taking position $D$
\item We refer to the candidate hedge by Portfolio $\theta = (\theta_0, \theta_1, \ldots, \theta_m)$ as the holdings in the fundamental assets
\item Our goal is to solve for the appropriate values of $x$ and $\theta$
\end{itemize}
\end{frame}

\begin{frame}
\frametitle{Appendix 4: Maximization of Expected Utility}
\begin{itemize}
\item Consider Utility of the combination of $D, -x, \theta$ in state $i$ at $t=1$:
$$U(V_D^{(i)} -x + \sum_{j=0}^m \theta_j \cdot (S_j^{(i)} - S_j^{(0)}))$$
\item So, the Expected Utility $f(V_D,x, \theta)$ at $t=1$ is given by:
$$f(V_D, x, \theta) = \sum_{i=1}^n \mu(\omega_i) \cdot U(V_D^{(i)} -x + \sum_{j=0}^m \theta_j \cdot (S_j^{(i)} - S_j^{(0)})) $$
\item Find $\theta$ that maximizes $f(V_D,x,\theta)$ with balance constraint at $t=0$
$$\max_{\theta} f(V_D, x, \theta) \mbox{ such that } x = - \sum_{j=0}^m \theta_j \cdot S_j^{(0)}$$
\end{itemize}
\end{frame}

\begin{frame}
\frametitle{Appendix 4: Maximization of Expected Utility}
\begin{itemize}
\item Re-write as unconstrained optimization (over $\theta' = (\theta_1, \ldots, \theta_m)$)
$$\max_{\theta'} g(V_D,x,\theta')$$
$$ \mbox{ where } g(V_D, x, \theta') = \sum_{i=1}^n \mu(\omega_i) \cdot U(V_D^{(i)} - x \cdot e^r + \sum_{j=1}^m \theta_j \cdot (S_j^{(i)} - e^r \cdot S_j^{(0)}))$$
\item {\em Price} of $D$ is defined as the ``breakeven value'' $z$ such that:
$$\sup_{\theta'} g(V_D, z, \theta') = \sup_{\theta'} g(0, 0, \theta')$$
\item Principle: Introducing a position of $V_D$ together with a cash receipt of $-z$ keeps the Maximum Expected Utility unchanged
\item $(\theta_1^*, \ldots, \theta_m^*)$ that achieves $\sup_{\theta'} g(V_D, z, \theta')$ and $\theta_0^* = -(z  + \sum_{j=1}^m \theta_j^* \cdot S_j^{(0)})$ are the associated hedges
\item Note that the Price of $V_D$ will NOT be the negative of the Price of $-V_D$, hence these prices serve as bounds/bid-ask prices
\end{itemize}
\end{frame}



\end{document}
