%%%%%%%%%%%%%%%%%%%%%%%%%%%%%%%%%%%%%%%%%
% Beamer Presentation
% LaTeX Template
% Version 1.0 (10/11/12)
%
% This template has been downloaded from:
% http://www.LaTeXTemplates.com
%
% License:
% CC BY-NC-SA 3.0 (http://creativecommons.org/licenses/by-nc-sa/3.0/)
%
%%%%%%%%%%%%%%%%%%%%%%%%%%%%%%%%%%%%%%%%%

%----------------------------------------------------------------------------------------
%	PACKAGES AND THEMES
%----------------------------------------------------------------------------------------

\documentclass[handout]{beamer}

\mode<presentation> {

% The Beamer class comes with a number of default slide themes
% which change the colors and layouts of slides. Below this is a list
% of all the themes, uncomment each in turn to see what they look like.

%\usetheme{default}
%\usetheme{AnnArbor}
%\usetheme{Antibes}
%\usetheme{Bergen}
%\usetheme{Berkeley}
%\usetheme{Berlin}
%\usetheme{Boadilla}
%\usetheme{CambridgeUS}
%\usetheme{Copenhagen}
%\usetheme{Darmstadt}
%\usetheme{Dresden}
%\usetheme{Frankfurt}
%\usetheme{Goettingen}
%\usetheme{Hannover}
%\usetheme{Ilmenau}
%\usetheme{JuanLesPins}
%\usetheme{Luebeck}
\usetheme{Madrid}
%\usetheme{Malmoe}
%\usetheme{Marburg}
%\usetheme{Montpellier}
%\usetheme{PaloAlto}
%\usetheme{Pittsburgh}
%\usetheme{Rochester}
%\usetheme{Singapore}
%\usetheme{Szeged}
%\usetheme{Warsaw}

% As well as themes, the Beamer class has a number of color themes
% for any slide theme. Uncomment each of these in turn to see how it
% changes the colors of your current slide theme.

%\usecolortheme{albatross}
%\usecolortheme{beaver}
%\usecolortheme{beetle}
%\usecolortheme{crane}
%\usecolortheme{dolphin}
%\usecolortheme{dove}
%\usecolortheme{fly}
%\usecolortheme{lily}
%\usecolortheme{orchid}
%\usecolortheme{rose}
%\usecolortheme{seagull}
%\usecolortheme{seahorse}
%\usecolortheme{whale}
%\usecolortheme{wolverine}

%\setbeamertemplate{footline} % To remove the footer line in all slides uncomment this line
%\setbeamertemplate{footline}[page number] % To replace the footer line in all slides with a simple slide count uncomment this line

%\setbeamertemplate{navigation symbols}{} % To remove the navigation symbols from the bottom of all slides uncomment this line
}

\usepackage{graphicx} % Allows including images
\usepackage{booktabs} % Allows the use of \toprule, \midrule and \bottomrule in tables
\usepackage{cool}
\usepackage{amsmath}
\usepackage{amssymb}
\usepackage{bm}
\usepackage{physics}
\usepackage{hyperref}
\usepackage{listings}
\usepackage{enumerate}

\newcommand{\prob}{\mathcal{P}}
\newcommand{\rew}{\mathcal{R}}
\newcommand{\states}{\mathcal{S}}
\newcommand{\actions}{\mathcal{S}}

\DeclareMathOperator*{\argmin}{arg\,min}
\DeclareMathOperator*{\argmax}{arg\,max}

\newcommand{\bvpi}{\bm{V}^{\pi}}
\newcommand{\bvs}{\bm{V}^*}
\newcommand{\bbpi}{\bm{B}^{\pi}}
\newcommand{\bbs}{\bm{B}^*}
\newcommand{\bv}{\bm{V}}

%----------------------------------------------------------------------------------------
%	TITLE PAGE
%----------------------------------------------------------------------------------------

\title[Approximate DP Chapter]{A Guided Tour of \href{http://stanford.edu/~ashlearn/RLForFinanceBook/book.pdf}{\underline{\textcolor{yellow}{Chapter 4}}}: \\ Function Approximation and Approximate DP} % The short title appears at the bottom of every slide, the full title is only on the title page

\author{Ashwin Rao} % Your name
\institute[Stanford] % Your institution as it will appear on the bottom of every slide, may be shorthand to save space
{ICME, Stanford University
 % Your institution for the title page
}

\date % Date, can be changed to a custom date

\begin{document}
\lstset{language=Python}  
\begin{frame}
\titlepage % Print the title page as the first slide
\end{frame}

% \begin{frame}
% \frametitle{Overview} % Table of contents slide, comment this block out to remove it
% \tableofcontents % Throughout your presentation, if you choose to use \section{} and \subsection{} commands, these will automatically be printed on this slide as an overview of your presentation
% \end{frame}

\begin{frame}
\frametitle{From DP to Approximate ADP (abbrev. ADP)}
\begin{itemize}[<+->]
\item Dynamic Programming algorithms meant for non-large finite spaces
\item DP algorithms typically sweep through all states in each iteration
\item Cannot do this for large finite spaces or for infinite spaces
\item Requires us to generalize to function approximation of Value Function
\begin{itemize}[<+->]
\item Sample an appropriate subset of states 
\item Calculate the Value Function for those states (Bellman calculation)
\item Create/Update a func approx with the sampled states' calculated values
\end{itemize}
\item Also, can sample transitions to estimate DP algo's Bellman update
\item The fundamental structure of the algorithms is still the same
\item Fundamental principles (Fixed-Point/Bellman Operators) still same
\item These generalizations known as {\em Approximate Dynamic Programming}
 \end{itemize}
\end{frame}

\begin{frame}
\frametitle{Theory of Function Approximations}
\pause
\begin{itemize}[<+->]
\item We work with a generic but simple setting for Function Approximation
\item Predictor variable $x \in \mathcal{X}$ (generic domain), Response variable $y \in \mathbb{R}$
\item We treat $x, y$ as unknown random variables, and want to estimate a function approximation for $\mathbb{P}[y|x]$ from data in the form of $(x,y)$ pairs
\item We consider parameterized functions $f$ with parameters denoted $w$
\item  Exact data type of $w$ will depend on specific form of function approx
\item  Denote the estimated probability of $y|x$ as $f(x; w)(y)$
\item  Assume given data in the form of a sequence of $n$ $(x,y)$ pairs:
$$[(x_i, y_i)|1 \leq i \leq n]$$
\item  Estimating $\mathbb{P}[y|x]$ is formalized by solving for $w=w^*$ such that:
$$w^* = \argmax_w \{ \prod_{i=1}^n f(x_i; w)(y_i)\} = \argmax_w \{ \sum_{i=1}^n \log f(x_i; w)(y_i)\}$$
\end{itemize}
\end{frame}

\begin{frame}
\frametitle{Maximum Likelihood and Cross-Entropy Loss}
\pause
\begin{itemize}[<+->]
\item This is the framework of {\em Maximum Likelihood Estimation} of $y|x$
\item Data $[(x_i, y_i)|1 \leq i \leq n]$ specifies {\em empirical probability distribution} $D$
\item Parameterized function $f$ specifies {\em model probability distribution} $M$
\item So we are in the business of reconciling $D$ and $M$
\item So this is minimizing {\em Cross-Entropy Loss} between $D$ and $M$
$$ \text{Cross-Entropy Loss } \mathcal{H}(D, M) = -\mathbb{E}_D[\log M]$$
\item We want to allow for incremental estimation (with data at each $t$):
$$[(x_{t,i}, y_{t,i})|1 \leq i \leq n_t]$$
\item Parameters update from $w_{t-1}$ to $w_t$ with say gradient descent
\item Allow for full batch, mini-batch or single pair (eg: SGD)
\item With an estimate of $f(x;w)$, we can predict $y|x$ as $\mathbb{E}_M[y|x]$:
$$\mathbb{E}_M[y|x] = \mathbb{E}_{f(x;w)}[y] = \int_{-\infty}^{+\infty} y \cdot f(x;w)(y) \cdot dy$$
\end{itemize}
\end{frame}


\begin{frame}[fragile]
\frametitle{The @abstractclass FunctionApprox}
\pause
\begin{lstlisting}
class FunctionApprox(ABC, Generic[X]):

    @abstractmethod
    def solve(
        self,
        xy_vals_seq: Iterable[Tuple[X, float]],
        error_tolerance: Optional[float] = None
    ) -> FunctionApprox[X]:
    
    @abstractmethod
    def evaluate(
        self,
        x_values_seq: Iterable[X]
    ) -> np.ndarray:
\end{lstlisting}
\end{frame}

\begin{frame}[fragile]
\frametitle{The @abstractclass FunctionApprox}
\pause
\begin{lstlisting}
    @abstractmethod
    def update(
        self,
        xy_vals_seq: Iterable[Tuple[X, float]]
    ) -> FunctionApprox[X]:
        pass
        
    def iterate_updates(
        self,
        xy_seq_stream: Iterator[Iterable[Tuple[X, float]]]
    ) -> Iterator[FunctionApprox[X]]:
        return iterate.accumulate(
            xy_seq_stream,
            lambda fa, xy: fa.update(xy),
            initial=self
        )   
\end{lstlisting}
\end{frame}


\begin{frame}
\frametitle{Linear Function Approximation}
\pause
\begin{itemize}[<+->]
\item Define a sequence of feature functions $\phi_j: \mathcal{X} \rightarrow \mathbb{R}, j = 1, 2, \ldots, m$
\item Parameters $w$ is a weights vector $\bm{w} = (w_1, w_2, \ldots, w_m) \in \mathbb{R}^m$
\item Linear function approximation assumes gaussian distribution for $y|x$
$$\text{ with mean } = \sum_{j=1}^m \phi_j(x) \cdot w_j \text{ and constant variance } \sigma^2$$
$$\mathbb{P}[y|x] = f(x;\bm{w})(y) = \frac {1} {\sqrt{2\pi \sigma^2}} \cdot e^{-\frac {(y - \sum_{j=1}^m \phi_j(x) \cdot w_j)^2} {2\sigma^2}}$$
\item Regularized cross-entropy loss function for data $[x_i, y_i|1 \leq i \leq n]$:
$$\mathcal{L}(\bm{w}) = \frac 1 {2n} \cdot \sum_{i=1}^n (\sum_{j=1}^m \phi_j(x_i) \cdot w_j - y_i)^2 + \frac 1 2 \cdot\lambda \cdot \sum_{j=1}^m w_j^2$$
\item This ignores constants involving $\sigma$, and $\lambda$ is regularization coefficient
\item So $\mathcal{L}(\bm{w})$ is just MSE of linear predictions $\sum_{j=1}^m \phi_j(x_i) \cdot w_j$
\end{itemize}
\end{frame}

\begin{frame}
\frametitle{Linear Function Approximation with Gradient Descent}
\pause
\begin{itemize}[<+->]
\item Gradient of $\mathcal{L}(\bm{w})$ with respect to $\bm{w}$ works out to:
$$\nabla_{\bm{w}} \mathcal{L}(\bm{w}) = \frac 1 n \cdot (\sum_{i=1}^n \bm{\phi}(x_i) \cdot (\bm{\phi}(x_i) \cdot \bm{w} - y_i)) + \lambda \cdot \bm{w}$$
$$\bm{\phi}(x) = (\phi_1(x), \phi_2(x), \ldots, \phi_m(x)) \text{ for all } x \in \mathcal{X}$$
\item Solve for $\bm{w^*}$ by incremental estimation using gradient descent
\item Gradient estimate $\mathcal{G}_{(x_t,y_t)}(\bm{w}_t)$ for time $t$-data $[(x_{t,i}, y_{t,i})|1 \leq i \leq n_t]$:
$$\mathcal{G}_{(x_t, y_t)}(\bm{w}_t) = \frac 1 n \cdot (\sum_{i=1}^{n_t} \bm{\phi}(x_{t,i}) \cdot (\bm{\phi}(x_{t,i}) \cdot \bm{w}_t - y_{t,i})) + \lambda \cdot \bm{w}_t$$
\item Interpreted as the weighted-mean of the feature vectors $\bm{\phi}(x_{t,i})$
\item Weighted by the (scalar) linear prediction errors $\bm{\phi}(x_{t,i}) \cdot \bm{w}_t - y_{t,i}$
\item So the update to the weights vector $\bm{w}$ is given by:
$$\bm{w}_{t+1} = \bm{w}_t - \alpha_t \cdot \mathcal{G}_{(x_t, y_t)}(\bm{w}_t)$$
\end{itemize}
\end{frame}

\begin{frame}
\frametitle{Direct Solution of Linear Function Approximation}
\pause
\begin{itemize}[<+->]
\item If feature functions not too large, we can directly solve for $\bm{w^*}$ 
\item Assume the entire provided data is $[(x_i, y_i)|1\leq i \leq n]$
\item Then the gradient estimate based can be set to 0 to solve for $\bm{w^*}$
$$\frac 1 n \cdot (\sum_{i=1}^n \bm{\phi}(x_i) \cdot (\bm{\phi}(x_i) \cdot \bm{w^*} - y_i)) + \lambda \cdot \bm{w^*} = 0$$
\item Denote $\bm{\Phi}$ as $n \times m$ matrix: $\bm{\Phi}_{i,j} = \phi_j(x_i)$
\item Denote column vector $\bm{Y} \in \mathbb{R}^n$ defined as $\bm{Y}_i = y_i$
$$\frac 1 n \cdot \bm{\Phi}^T \cdot (\bm{\Phi} \cdot \bm{w^*} - \bm{Y}) + \lambda \cdot \bm{w^*} = 0$$
$$\Rightarrow (\bm{\Phi}^T \cdot \bm{\Phi} + n \lambda \cdot \bm{I_m}) \cdot \bm{w^*} = \bm{\Phi}^T \cdot \bm{Y}$$
$$\Rightarrow \bm{w^*} = (\bm{\Phi}^T \cdot \bm{\Phi} + n \lambda \cdot \bm{I_m})^{-1} \cdot \bm{\Phi}^T \cdot \bm{Y}$$
\end{itemize}
\end{frame}

\begin{frame}
\frametitle{Deep Neural Networks (of vanilla flavor)}
\pause
\begin{itemize}[<+->]
\item Deep Neural Network (DNN) layers numbered $l = 0, 1, \ldots L$
\item Denote input and output to layer $l$ as vectors $\bm{I_l}$ and $\bm{O_l}$
$$\bm{I_0} = \bm{\phi}(x) \in \mathbb{R}^m \text{ and } \bm{O_L} = \mathbb{E}_M[y|x] \text{ and } \bm{I_{l+1}} = \bm{O_l}$$
\item Denote layer $l$ parameters as $|\bm{O_l}| \times |\bm{I_l}|$ matrix $\bm{w_l}$
\item Layer $l$ neurons define a linear transformation from $\bm{I_l}$ to $\bm{S_l}$
$$\bm{S_l} = \bm{w_l} \cdot \bm{I_l} \text{ and } \bm{O_l} = g_l(\bm{S_l})$$
\item where $g_l: \mathbb{R} \rightarrow \mathbb{R}$ is the activation function for layer $l$
\item Forward-propagation composes layers' linear and activation functions
\item Back-propagation calculates cross-entropy loss gradient $\nabla_{\bm{w_l}} \mathcal{L}$
\item Gradient Descent updates for weights $\bm{w_l}$ proportional to $\nabla_{\bm{w_l}} \mathcal{L}$
\end{itemize}
\end{frame}

\begin{frame}
\frametitle{Back-prop as Recursive Gradient Calculation}
\pause
\begin{itemize}[<+->]
\item Loss gradient can be reduced to calculating $\bm{P_l} = \nabla_{\bm{S_l}} \mathcal{L}$
$$\nabla_{\bm{w_l}} \mathcal{L} = (\nabla_{\bm{S_l}} \mathcal{L})^T \cdot \nabla_{\bm{w_l}} \bm{S_l} = \bm{P_l}^T \cdot \nabla_{\bm{w_l}} \bm{S_l} = \bm{P_l} \cdot \bm{I_l}^T = \bm{P_l} \otimes \bm{I_l}$$
\item Including $L^2$ regularization (with regularization coefficients $\lambda_l$):
$$\nabla_{\bm{w_l}} \mathcal{L} = \bm{P_l} \otimes \bm{I_l} + \lambda_l \cdot \bm{w_l} $$
\item $\cdot$ is inner-product, $\otimes$ is outer-product, $\circ$ is component-wise product
\end{itemize}
\pause
\begin{theorem}
$$\bm{P_l} = (\bm{w_{l+1}}^T \cdot \bm{P_{l+1}}) \circ g_l'(\bm{S_l}) \text{ (read the proof in the book)}$$
\end{theorem}
\pause
To calculate $\bm{P_L} = \nabla_{\bm{S_L}} \mathcal{L}$, assume suitable functional form for $\mathbb{P}[y|S_L]$
\end{frame}

\begin{frame}
\frametitle{Exponential functional form for $\mathbb{P}[y|S_L]$}
\pause
\begin{itemize}[<+->]
\item Consider the exponential-family functional-form for $\mathbb{P}[y|S_L]$
$$\mathbb{P}[y|S_L] = p(y|S_L, \tau) = h(y, \tau) \cdot e^{\frac {S_L \cdot y - A(S_L)} {d(\tau)}}$$
\item Form adopted from framework of Generalized Linear Models (GLM)
\item We want the scalar prediction $O_L = g_L(S_L)$ to be equal to $\mathbb{E}_p[y|S_L]$
\item What function $g_L: \mathbb{R} \rightarrow \mathbb{R}$ (in terms of $p(y|S_L, \tau)$) would satisfy the requirement of $O_L = g_L(S_L) = \mathbb{E}_p[y|S_L]$?
\end{itemize}
\pause
\begin{lemma}
$$\mathbb{P}[y|S_L] = h(y, \tau) \cdot e^{\frac {S_L \cdot y - A(S_L)} {d(\tau)}} \Rightarrow \mathbb{E}_p[y|S_L] = A'(S_L)$$
\end{lemma}
\pause
\begin{itemize}[<+->]
\item To satisfy $O_L = g_L(S_L) = \mathbb{E}_p[y|S_L]$, we need: $O_L = g_L(S_L) = A'(S_L)$
\item So $g_L(\cdot)$ must be set to be the derivative of the $A(\cdot)$ function
\item In GLM theory, $A'(\cdot)$ serves as {\em canonical link function} for given $\mathbb{P}[y|x]$
\end{itemize}
\end{frame}

\begin{frame}
\frametitle{Examples of Distributions and their Canonical Links}
With canonical link, $P_L$ reduces to prediction error for each $(x,y)$ data
\begin{theorem}
$$P_L =  \frac {\partial \mathcal{L}}{\partial S_L} = \frac {O_L - y} {d(\tau)}$$
\end{theorem}
\pause
Some examples of distributions and their canonical link functions:
\begin{itemize}
\item Normal distribution $y \sim \mathcal{N}(\mu, \sigma^2)$: $S_L = \mu, \tau = \sigma, h(y, \tau) = \frac {e^{\frac {-y^2} {2 \tau^2}}} {\sqrt{2 \pi} \tau}, A(S_L) = \frac {S_L^2} {2}, d(\tau) = \tau^2$. $g_L(S_L) = S_L$
\item Bernoulli distribution for binary-valued $y$, parameterized by $p$: $S_L = \log{(\frac p {1-p})}, \tau = h(y, \tau) = d(\tau) = 1, A(S_L) = \log{(1+e^{S_L})}$. $g_L(S_L) = \frac 1 {1+e^{-S_L}}$
\item Poisson distribution for $y$ parameterized by $\lambda$:
$S_L = \log{\lambda}, \tau = d(\tau) = 1, h(y, \tau) = \frac 1 {y!}, A(S_L) = e^{S_L}$.  $g_L(S_L) = e^{S_L}$
\end{itemize}
\end{frame}

\begin{frame}
\frametitle{Tabular as a form of {\em FunctionApprox}}
\pause
\begin{itemize}[<+->]
\item ``Tabular'' is simple setting with finite $\mathcal{X} = \{x_1, x_2, \ldots, x_n\}$
\item With $(x,y)$ data pairs having all it's $x$-values within this finite $\mathcal{X}$
\item $\mathbb{E}[y|x]$ must be calculated from data $y$-values associated with single $x$
\item So $\mathbb{E}[y|x]$ prediction must be {\em some sort of average} of those $y$-values
\item A ``table'' can store all $\mathcal{X}$ together with all predictions $\mathbb{E}[y|x]$
\item This ``Tabular'' setting is compatible with \lstinline{FunctionApprox} interface
\item Also, ``Tabular'' is a special case of linear function approximation
\item With features $\phi_i$ as indicator functions for each $x_i \in \mathcal{X}$
\item And weights $w_i$ as average of $y$-values associated with $x_i$ in the data
\item Next we cover Approximate DP using \lstinline{FunctionApprox}
\item Where $\mathcal{X}$ is state space and predictions constitute Value Function
\item Specializing \lstinline{FunctionApprox} to \lstinline{Tabular} gives Tabular DP
\end{itemize}
\end{frame}


\begin{frame}
\frametitle{Approximate Policy Evaluation}
\pause
\begin{itemize}[<+->]
\item Repeatedly apply $\bm{B}^{\pi}$ on \lstinline{FunctionApprox} of $V: \mathcal{N} \rightarrow \mathbb{R}$	
\item Operates on \lstinline{MarkovRewardProcess} (not necessarily Finite)
\item So no enumeration of states and no access to transition probabilities
\item We specify a sampling probability distribution of ``source states''
\item From each source sample $s$, sample pairs of (next state $s'$, reward $r$)
\item Estimate $\mathbb{E}[r + \gamma \cdot V(s')]$ by averaging over sampled pairs
\item  $V(s')$ obtained from the instance of \lstinline{FunctionApprox} being used
\item Sample of source states and their associated Bellman expectation estimates (from transition samples) used to \lstinline{update} \lstinline{FunctionApprox}
\end{itemize}
\end{frame}

\begin{frame}[fragile]
\frametitle{Approximate Policy Evaluation interface}
\pause
\begin{lstlisting}
def evaluate_mrp(
    mrp: MarkovRewardProcess[S],
    gamma: float,
    approx_0: FunctionApprox[S],
    non_terminal_states_distribution: Distribution[S],
    num_state_samples: int
) -> Iterator[FunctionApprox[S]]:
\end{lstlisting}
\end{frame}

\begin{frame}[fragile]
\frametitle{Approximate Policy Evaluation code}
\pause
\begin{lstlisting}
def update(v: FunctionApprox[S]) -> FunctionApprox[S]:
    nt_states: Sequence[S] = \
        non_terminal_states_distribution.sample_n(
            num_state_samples
        )

    def return_(s_r: Tuple[S, float]) -> float:
        s, r = s_r
        return r + gamma * v.evaluate([s]).item()

    return v.update(
        [(s, mrp.transition_reward(s).expectation(
            return_)) for s in nt_states]
    )

return iterate(update, approx_0)
\end{lstlisting}
\end{frame}

\begin{frame}[fragile]
\frametitle{Approximate Value Iteration interface}
\pause
\begin{lstlisting}
def value_iteration(
    mdp: MarkovDecisionProcess[S, A],
    gamma: float,
    approx_0: FunctionApprox[S],
    non_terminal_states_distribution: Distribution[S],
    num_state_samples: int
) -> Iterator[FunctionApprox[S]]:
\end{lstlisting}
\end{frame}

\begin{frame}[fragile]
\frametitle{Approximate Value Iteration code}
\pause
\begin{lstlisting}
def update(v: FunctionApprox[S]) -> FunctionApprox[S]:
    nt_states: Sequence[S] = \
        non_terminal_states_distribution.sample_n(
            num_state_samples
        )

    def return_(s_r: Tuple[S, float]) -> float:
        s, r = s_r
        return r + gamma * v.evaluate([s]).item()

    return v.update(
        [(s, max(mdp.step(s, a).expectation(return_)
                      for a in mdp.actions(s)))
         for s in nt_states]
    )
return iterate(update, approx_0)
\end{lstlisting}
\end{frame}



\begin{frame}
\frametitle{Finite-Horizon Approximate Dynamic Programming}
\pause
\begin{itemize}[<+->]
\item Similarly, generalize Backward Induction DP algorithms
\item Each time steps' Value Function is a \lstinline{FunctionApprox}
\item Work with a separate MRP/MDP representation for each time step's transitions, that is responsible for sampling next step's (state, reward)
\item $x$-values come from current time step's states sampling distribution
\item $y$-values come from applying Bellman Operator on next time steps' \lstinline{FunctionApprox} for it's Value Function
\item  Bellman Operator expectation is estimated by averaging over transition samples
\item These $(x,y)$ pairs constitute the data-set used to \lstinline{solve} the current time step's \lstinline{FunctionApprox} for it's Value Function
\end{itemize}
\end{frame}

\begin{frame}
\frametitle{Constructing the Non-Terminal States Distribution}
\pause
\begin{itemize}[<+->]
\item Each ADP algorithm works with a distribution of non-terminal states
\item Good choice: Stationary Distribution of uniform-policy-implied MRP
\item See if you can use some mathematical property of given MDP/MRP
\item Or create sampling traces and estimate with occurrence frequency
\item Backup choice: Uniform Distribution of all non-terminal states
\item Likewise, for backward induction, see if you can utilize some property of the given process to infer distribution of states for a fixed time step
\item eg: In finance, continuous-time processes can sometimes be solved
\item Or create sampling traces and estimate with occurrence frequency
\item Backup choice: Uniform Distribution of all non-terminal states
\end{itemize}
\end{frame}


\begin{frame}
\frametitle{Key Takeaways from this Chapter}
\pause
\begin{itemize}[<+->]
\item The \lstinline{FunctionApprox} interface involves three key methods:
\begin{itemize}
\item \lstinline{solve:} Calculate the ``best-fit'' parameters that minimizes the cross-entropy loss function for the given fixed data set of $(x,y)$ pairs
\item \lstinline{update:} Parameters of \lstinline{FunctionApprox} are updated based on each new $(x,y)$ pairs data set from the available data stream
\item \lstinline{evaluate:}  Calculate the conditional expectation of response variable $y$, according to the model specified by \lstinline{FunctionApprox}
\end{itemize}
\item Tabular is a special case of linear function approximation with feature functions as indicator functions for each of the finite set of $\mathcal{X}$
\item All the Tabular DP algorithms can be generalized to ADP algorithms
\begin{itemize}
\item Tabular VF updates replaced by updates to \lstinline{FunctionApprox} parameters
\item Sweep over all states in Tabular case replaced by state samples
\item Bellman Operators' Expectation estimated as average of calculations over transition samples (versus using explicit transition probabilities)
\end{itemize}
\end{itemize}
\end{frame}


\end{document}