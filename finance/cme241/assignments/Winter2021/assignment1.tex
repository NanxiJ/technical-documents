\documentclass[12pt]{exam}
\usepackage[utf8]{inputenc}
\usepackage{graphicx} % Allows including images
\usepackage{cool}
\usepackage{tikz}
\usepackage{amsmath}
\usepackage{listings}
\usepackage{pseudocode}
\usepackage[colorlinks = true,
            linkcolor = blue,
            urlcolor  = blue,
            citecolor = blue,
            anchorcolor = blue]{hyperref}
\usepackage{MnSymbol,wasysym}
\usepackage{geometry} % see geometry.pdf on how to lay out the page. There's lots.
\geometry{a4paper} 
\newgeometry{vmargin={20mm}, hmargin={14mm,18mm}}
 
\begin{document}
\begin{center}
{\large {\bf Stanford CME 241 (Winter 2021) - Assignment 1}}
\end{center}

I trust that you will follow \href{https://communitystandards.stanford.edu/policies-and-guidance/honor-code}{The Stanford Honor Code}.
 
{\large{\bf Assignments:}}
\begin{questions}
\question Register for the \href{https://piazza.com/stanford/winter2021/cme241/home}{Course on Piazza}.
\question Install/Setup on your laptop with LaTeX, Python 3 (and optionally Jupyter notebook).
\question Create a git repo for this course where you can upload and organize all the code and technical writing you will do as part of assignments and self-learning.
\question Send Sven a message on Piazza with your git repo URL, so he can periodically review your assignments and other self-learning work.
\question Clone the \href{https://github.com/TikhonJelvis/RL-book}{Code Repo associated with the RLForFinanceBook} and get set up to write code (for future assignments that uses classes/functions from this code repo). {\bf Optionally}, you can create the same virtual environment I use and replicate my dependencies with the following instructions:
\begin{itemize}
\item After cloning the repo on your laptop, create a virtual environment with the following shell command (from the RL-Book directory):
\begin{lstlisting}[language=bash]
$ python3 -m venv .venv
\end{lstlisting}
\item Then, each time you're working on this project, make sure to activate the venv with the following shell command (again, from the RL-Book directory):
\begin{lstlisting}[language=bash]
$ source .venv/bin/activate
\end{lstlisting}
\item Once the venv is activated, you should see a (.venv) in your shell prompt
\item Now you can use pip to install dependencies inside the venv, for example:
\begin{lstlisting}[language=bash]
(.venv) $ pip install matplotlib
\end{lstlisting}
\item Initially, you can install every Python package you need to work this git repo with the following shell command (again, from the RL-Book directory):
\begin{lstlisting}[language=bash]
(.venv) $ pip install -r requirements.txt
\end{lstlisting}
\item To work with the appropriate file paths of the Python files in this repo from the RL-Book directory, execute the following command from the RL-book directory (this creates a pckage):
\begin{lstlisting}[language=bash]
(.venv) $ pip install -e .
\end{lstlisting}
\item To make sure you are all good, verify with the following command from the RL-book directory:
\begin{lstlisting}[language=bash]
(.venv) $ python -m unittest discover
\end{lstlisting}
If all is good, you should see an "OK" on the last line of the output upon running this command.
\end{itemize}
\end{questions}

\end{document}