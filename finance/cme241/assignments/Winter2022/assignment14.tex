\documentclass[12pt]{exam}
\usepackage[utf8]{inputenc}
\usepackage{graphicx} % Allows including images
\usepackage{cool}
\usepackage{tikz}
\usepackage{amsmath}
\usepackage{listings}
\usepackage{pseudocode}
\usepackage[colorlinks = true,
            linkcolor = blue,
            urlcolor  = blue,
            citecolor = blue,
            anchorcolor = blue]{hyperref}
\usepackage{MnSymbol,wasysym}
\usepackage{geometry} % see geometry.pdf on how to lay out the page. There's lots.
\geometry{a4paper} 
\newgeometry{vmargin={20mm}, hmargin={14mm,18mm}}
 
\begin{document}
\begin{center}
{\large {\bf Stanford CME 241 (Winter 2022) - Assignment 14}}
\end{center}
 
{\large{\bf Assignments:}}

{\bf Do 1 of \{1,2\} and the corresponding of \{3,4\}}

\begin{questions}

\question Implement in Python the LSTD Algorithm, as covered in class.
\question Implement in Python the LSPI Algorithm, as covered in class.
\question Implement in Python LSPI customized for American Options Pricing, as covered in class. Test by comparing the pricing of American Calls and Puts against the Binomial Tree implementation in \href{https://github.com/TikhonJelvis/RL-book/blob/master/rl/chapter8/optimal_exercise_bin_tree.py}{rl\//chapter8\//optimal\_exercise\_bin\_tree.py}.
\question Implement in Python Deep Q-Learning customized for American Options Pricing, as covered in class. Test by comparing the pricing of American Calls and Puts against the Binomial Tree implementation in \href{https://github.com/TikhonJelvis/RL-book/blob/master/rl/chapter8/optimal_exercise_bin_tree.py}{rl\//chapter8\//optimal\_exercise\_bin\_tree.py}.

\end{questions}

\end{document}