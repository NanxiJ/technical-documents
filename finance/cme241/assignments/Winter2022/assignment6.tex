\documentclass[12pt]{exam}
\usepackage[utf8]{inputenc}
\usepackage{graphicx} % Allows including images
\usepackage{cool}
\usepackage{tikz}
\usepackage{amsmath}
\usepackage{listings}
\usepackage{pseudocode}
\usepackage[colorlinks = true,
            linkcolor = blue,
            urlcolor  = blue,
            citecolor = blue,
            anchorcolor = blue]{hyperref}
\usepackage{MnSymbol,wasysym}
\usepackage{geometry} % see geometry.pdf on how to lay out the page. There's lots.
\geometry{a4paper} 
\newgeometry{vmargin={20mm}, hmargin={14mm,18mm}}
 
\begin{document}
\begin{center}
{\large {\bf Stanford CME 241 (Winter 2022) - Assignment 6}}
\end{center}
 
{\large{\bf Assignments:}}
\begin{questions}
\question Assume the Utility function is $U(x) = x - \frac {\alpha x^2} 2$. Assuming $x \sim \mathcal{N}(\mu, \sigma^2)$, calculate:
\begin{itemize}
\item Expected Utility $\mathbb{E}[U(x)]$
\item Certainty-Equivalent Value $x_{CE}$
\item Absolute Risk-Premium $\pi_A$	
\end{itemize}
Assume you have a million dollars to invest for a year and you are allowed to invest $z$ dollars in a risky asset whose annual return on investment is $\mathcal{N}(\mu, \sigma^2)$ and the remaining (a million minus $z$ dollars) would need to be invested in a riskless asset with fixed annual return on investment of $r$. You are not allowed to adjust the quantities invested in the risky and riskless assets after your initial investment decision at time $t=0$ (static asset allocation problem). If your risk-aversion is based on this Utility function, how much would you invest in the risky asset? In other words, what is the optimal value for $z$, given your level of risk-aversion (determined by a fixed value of $\alpha$)?

Plot how the optimal value of $z$ varies with $\alpha$.

\question {\bf Optional:} Repeat the calculations for the {\em Portfolio application of CRRA} (that we covered in class) with a Utility function of $U(x) = \log(x)$ (instead of $U(x) = \frac {x^{1 - \gamma} - 1} {1 - \gamma}$).

\question Assume you are playing a casino game where at every turn, if you bet a quantity $x$, you will be returned $x \cdot (1 + \alpha)$ with probability $p$ and returned $x \cdot (1 - \beta)$ with probability $q = 1 - p$ for $\alpha, \beta \in \mathbb{R}^+$ (i.e., the return on bet is $\alpha$ with probability $p$ and $-\beta$ with probability $q = 1-p$) . The problem is to identify a betting strategy that will maximize one's expected wealth over the long run. The optimal solution to this problem is known as the Kelly criterion, which involves betting a constant fraction of one's wealth at each turn (let us denote this optimal fraction as $f^*$).

It is known that the Kelly criterion (formula for $f^*$) is equivalent to maximizing the Expected Utility of Wealth after a single bet, with the Utility function defined as: $U(W) = \log(W)$. Denote your wealth before placing the single bet as $W_0$. Let $f$ be the fraction (to be solved for) of $W_0$ that you will bet. Therefore, your bet is $f \cdot W_0$.

\begin{itemize}
\item Write down the two outcomes for wealth $W$ at the end of your single bet of $f \cdot W_0$.
\item Write down the two outcomes for $\log$ (Utility) of $W$.
\item Write down $\mathbb{E}[\log(W)]$.
\item Take the derivative of $\mathbb{E}[\log(W)]$ with respect to $f$. 
\item Set this derivative to 0 to solve for $f^*$. Verify that this is indeed a maxima by evaluating the second derivative at $f^*$. This formula for $f^*$ is known as the Kelly Criterion. 
\item Convince yourself that this formula for $f^*$ makes intuitive sense (in terms of it's dependency on $\alpha$, $\beta$ and $p$).
\end{itemize}

\end{questions}

\end{document}